% references with “namespaces”

\newcommand*{\tableref}[1]{Table~\ref{table:#1}}
\newcommand*{\figref}[1]{Fig.~\ref{fig:#1}}
\newcommand*{\secref}[1]{Sec.~\ref{sec:#1}}
\newcommand*{\defref}[1]{Def.~\ref{defn:#1}}
\newcommand*{\appref}[1]{App.~\ref{sec:#1}}%\suppref{}}

% useful abbreviations – currently written UK English style

\newcommand{\wrt}{{{w.r.t.\@}}}
\newcommand{\etc}{{{etc.\@}}}
\newcommand{\eg}{{{e.g.\@}}}
\newcommand{\resp}{{{resp.\@}}}
\newcommand{\ie}{{{i.e.\@}}}
\newcommand{\cf}{{{cf.\@}}}
\newcommand{\etal}{{{et~al.\@}}}


%% comments/todos/etc. with colour
\usepackage[normalem]{ulem}
\usepackage{color}

% uncomment \def\nocolour{ } to eliminate all text colour
% \def\nocolour{ }
% \def\sout#1{} % comment this to get the default strikethrough - this overrides the existing macro


\newcommand{\soutifcolour}[1]{\ifdefined\nocolour{}\else{\protect\sout{#1}}\fi}

\newcommand\todo[1]{\ifdefined\nocolour{}\else{\textcolor{red}{TODO: #1}}\fi}

\newcommand{\peter}[1]{\ifdefined\nocolour{#1}\else{\color{blue}{#1}}\fi}
\newcommand{\pfootnote}[1]{\ifdefined\nocolour{}\else\peter{\footnote{\peter{PETER: #1}}}\fi}
\newcommand{\pout}[1]{\peter{{\soutifcolour{#1}}}}

\newcommand{\as}[1]{\ifdefined\nocolour{{#1}}\else{\color{orange}{#1}}\fi}
\newcommand{\asfootnote}[1]{\ifdefined\nocolour{}\else\as{\footnote{\as{ALEX: #1}}}\fi}
\newcommand{\asout}[1]{\as{\soutifcolour{#1}}}


% OPERATORS: try e.g. \def\FV{\operator{\textit{FV}}}

\makeatletter
% Optional arguments - if followed by a { then generates ( ) around argument
\def\operator#1{\@ifnextchar\bgroup {\operatorarg{\ensuremath{#1}}}{\ensuremath{#1}}}
\def\operatorarg#1#2{{#1}{\ensuremath{(#2)}}}
% Optional arguments - if followed by a { then generates operator which takes #3 and produces #1#2#3
\def\spoperator#1#2{\@ifnextchar\bgroup{\spoperatorarg{\ensuremath{#1}}{\ensuremath{#2}}}{\ensuremath{#1}}}
\def\spoperatorarg#1#2#3{\ensuremath{#1#2#3}}

% Optional arguments - as above, but parses argument {A,B} and produces A.op(B)
% WARNING: ~ character should not occur in the arguments of this macro..
\def\fixedoperator#1{\@ifnextchar\bgroup {\fixedoperatorarg{#1}}{\ensuremath{#1}}}
\def\fixedoperatorarg#1#2{\fixedoperatorparse{#1}#2~}
\def\fixedoperatorparse#1#2,#3~{\ensuremath{{#2}{.}{#1}{(#3)}}}
\makeatother


% Inference rules - try \Inf{blah}{blah2}.{bleh}{bleeh} – requires the “prooftree” package (prooftree.sty)
\makeatletter

\newskip \point

\point =1pt

\def \premisespacing{\quad}

\def \RulePremisesNewlineMore[#1]#2.#3#4{\@ifnextchar\bgroup{\RulePremisesNewlineMore[#1]{#2}.{#3\premisespacing#4}}{\@ifnextchar.{\RulePremisesNewline[#1]{{\begin{array}{c}#2\\#3\premisespacing#4\end{array}}}}{\RuleMultiPremise[#1]{{\begin{array}{c}#2\\#3\end{array}}}{#4}}}}

\def \RulePremisesNewline[#1]#2.#3{\@ifnextchar\bgroup{\RulePremisesNewlineMore[#1]{#2}.{#3}}{\@ifnextchar.{\RulePremisesNewline[#1]{{\begin{array}{c}#2\\#3\end{array}}}}{\RuleMultiPremise[#1]{#2}{#3}}}}

\def \RuleMultiPremise[#1]#2#3{\@ifnextchar\bgroup{\RuleMultiPremise[#1]{#2\premisespacing#3}}{\@ifnextchar.{\RulePremisesNewline[#1]{#2\premisespacing#3}}{\prooftree #2\justifies#3 \using{#1}\endprooftree}}}

\def \RuleWithName[#1]#2{\@ifnextchar\bgroup {\RuleMultiPremise[#1]{#2}}{\@ifnextchar.{\RulePremisesNewline[#1]{#2}}{\prooftree \justifies #2 \using{#1} \endprooftree}}}

\def \RuleWithInfo[#1]{\@ifnextchar[{\RuleWithNameAndCondition[#1]}{\RuleWithName[(#1)]}}

\def \RuleWithNameAndCondition[#1][#2]{\RuleWithName[(#1)^{#2}]}

\def \Inf{\proofrulebaseline=2ex \abovedisplayskip12\point\belowdisplayskip12\point \abovedisplayshortskip8\point\belowdisplayshortskip8\point \@ifnextchar[{\RuleWithInfo}{\RuleWithName[ ]}}

\makeatother
